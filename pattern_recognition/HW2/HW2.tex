%!TEX program = xelatex
\documentclass[a4paper,UTF8]{article}
\usepackage{ctex}
\usepackage[margin=1.25in]{geometry}
\usepackage{color}
\usepackage{graphicx}
\usepackage{amssymb}
\usepackage{amsmath}
\usepackage{amsthm}
\usepackage{enumerate}
\usepackage{bm}
\usepackage{hyperref}
\usepackage{epsfig}
\usepackage{color}
\usepackage{mdframed}
\usepackage{lipsum}
\usepackage{graphicx}
\usepackage{float}
\newmdtheoremenv{thm-box}{Theorem}
\newmdtheoremenv{prop-box}{Proposition}
\newmdtheoremenv{def-box}{定义}

\usepackage{listings}
\usepackage{xcolor}
\lstset{
	numbers=left,
	numberstyle= \tiny,
	keywordstyle= \color{ blue!70},
	commentstyle= \color{red!50!green!50!blue!50},
	frame=shadowbox, % 阴影效果
	rulesepcolor= \color{ red!20!green!20!blue!20} ,
	escapeinside=``, % 英文分号中可写入中文
	xleftmargin=2em,xrightmargin=2em, aboveskip=1em,
	framexleftmargin=2em
}

\usepackage{booktabs}

\setlength{\evensidemargin}{.25in}
\setlength{\textwidth}{6in}
\setlength{\topmargin}{-0.5in}
\setlength{\topmargin}{-0.5in}
% \setlength{\textheight}{9.5in}
%%%%%%%%%%%%%%%%%%此处用于设置页眉页脚%%%%%%%%%%%%%%%%%%
\usepackage{fancyhdr}
\usepackage{lastpage}
\usepackage{layout}
\footskip = 12pt
\pagestyle{fancy}                    % 设置页眉
\lhead{2021年春季}
\chead{模式识别}
% \rhead{第\thepage/\pageref{LastPage}页}
\rhead{作业二}
\cfoot{\thepage}
\renewcommand{\headrulewidth}{1pt}  			%页眉线宽,设为0可以去页眉线
\setlength{\skip\footins}{0.5cm}    			%脚注与正文的距离
\renewcommand{\footrulewidth}{0pt}  			%页脚线宽,设为0可以去页脚线

\makeatletter 									%设置双线页眉
\def\headrule{{\if@fancyplain\let\headrulewidth\plainheadrulewidth\fi%
\hrule\@height 1.0pt \@width\headwidth\vskip1pt	%上面线为1pt粗
\hrule\@height 0.5pt\@width\headwidth  			%下面0.5pt粗
\vskip-2\headrulewidth\vskip-1pt}      			%两条线的距离1pt
 \vspace{6mm}}     								%双线与下面正文之间的垂直间距
\makeatother

%%%%%%%%%%%%%%%%%%%%%%%%%%%%%%%%%%%%%%%%%%%%%%
\numberwithin{equation}{section}
%\usepackage[thmmarks, amsmath, thref]{ntheorem}
\newtheorem{theorem}{Theorem}
\newtheorem*{definition}{Definition}
\newtheorem*{solution}{Solution}
\newtheorem*{prove}{Proof}
\newcommand{\indep}{\rotatebox[origin=c]{90}{$\models$}}

\usepackage{multirow}

%--

%--
\begin{document}
\title{模式识别\\
作业二}
\author{181220076, 周韧哲, 本科,人工智能学院,人工智能学院选课}
\maketitle

\section*{Problem 1}
\begin{enumerate}[(a)]
	\item $\|x_j-\mu_i\|^2$代表$x_j$到第$i$组的代表$u_i$的距离,当$\gamma_{ij}=0$时,这一项为$0$,说明$\sum_{i=1}^{K}\gamma_{ij}\|x_j-\mu_i\|^2$就是被分到了第$k$组的$x_j$到与其类代表的距离。从而,$\sum_{j=1}^{M}\sum_{i=1}^{K}\gamma_{ij}\|x_j-\mu_i\|^2$就代表所有的样本到其对应类代表的距离和。对其最小化也就是让相同组的样本到其类代表的距离和最小,即属于相同组的样本彼此相似,距离最小。所以该式形式化了K均值目标。
	\item 设第$k$次迭代后为$\gamma_{ij}^k,\mu_i^k$。
	\begin{enumerate}[i.]
		\item 固定$u_i$后,$\gamma_{ij}^{k+1} = \arg\min_{\gamma_{ij}^{k}} \sum_{i=1}^{K}\sum_{j=1}^{M}\gamma_{ij}^k\|x_j-\mu_i^k\|^2$。固定$\mu_i$后,上式等价于对于每个$x_j$,$\gamma_{ij}^{k+1} = \arg\min_{\gamma_{ij}^{k}} \sum_{i=1}^{K}\gamma_{ij}^k\|x_j-\mu_i^k\|^2$,即选择距离最小的$\mu_i^k$对应类别作为$x_j$的类别。所以
		$$\gamma_{ij}^{k+1} = \begin{cases}
			1,\quad\|x_j-\mu_j^{k}\|^2\leq \|x_j-\mu_{j'}^{k}\|^2,j'=1,2,\cdots,K \\
			0,\quad\text{otherwise}
		\end{cases}$$
		\item 固定$\gamma_{ij}$后,$\mu_i^{k+1} = \arg\min_{\mu_i^{k}} \sum_{i=1}^{K}\sum_{j=1}^{M}\gamma_{ij}^{k+1}\|x_j-\mu_i^k\|^2$。$\sum_{j=1}^{M}\gamma_{ij}^{k+1}\|x_j-\mu_i^k\|^2$就是第$i$类样本到$\mu_i^k$的距离和,因而固定$\gamma_{ij}$后可分别优化每个$\mu_i$:$\mu_i^{k+1}=\arg\min_{\mu_i^{k}} \sum_{j=1}^{M}\gamma_{ij}^{k+1}\|x_j-\mu_i^k\|^2$,由于$f(x)=\|x\|^2$为凸函数,故可求导等于0得到最小值$$u_i^{k+1}=\frac{\sum_{j=1}^{M}\gamma_{ij}^{k+1}x_j}{\sum_{j=1}^{M}\gamma_{ij}^{k+1}}$$
	\end{enumerate}
    \item 记目标函数为$J(\gamma,\mu)$,首先证明在Floyd算法中$J(\gamma,\mu)$递减。在$i$步,固定$\mu$后,由于对于每个样本均归类到离其最近的样本中心,即任意$x_j$都有$\|x_j-\mu_j^{k}\|^2\leq \|x_j-\mu_{j'}^{k}\|^2,j'=1,2,\cdots,K$,从而$J_{k+1}'(\gamma,\mu)\leq J_k(\gamma,\mu)$。在$ii$步,固定$\gamma$后,对于每个$\mu_i$,都取了$\sum_{j=1}^{M}\gamma_{ij}^{k+1}\|x_j-\mu_i^k\|^2$的最小值,从而,$J_{k+1}(\gamma,\mu)\leq J_{k+1}'(\gamma,\mu)$。又因为$J(\gamma,\mu)\geq 0$显然成立,从而目标函数单调递减且有界,必定收敛。
\end{enumerate}

\section*{Problem 2}
\begin{enumerate}[(a)]
	\item $\min_\beta\sum_{i=1}^{n}(y_i-x_i^T\beta)^2$
	\item $\min_\beta(y-X\beta)^T(y-X\beta)$
	\item 将优化项展开,得到$$\min_\beta y^Ty-y^TX\beta-\beta^TX^Ty+\beta^TX^TX\beta$$
	      由于中间两项为标量,所以可写为$\min_\beta y^Ty-2\beta^TX^Ty+\beta^TX^TX\beta$,其为二次规划问题,可直接求梯度等于0,得到$X^TX\beta=X^Ty$,假设$X^TX$可逆,所以$\beta^*=(X^TX)^{-1}X^Ty$。
	\item 不可逆,当$d>n$时$X^TX$为奇异矩阵。
	\item 该正则项会使得$\beta$是有偏估计。
	\item $\min_\beta(y-X\beta)^T(y-X\beta)+\lambda\beta^T\beta$,展开得到$$\min_\beta y^Ty-y^TX\beta-\beta^TX^Ty+\beta^T(X^TX+\lambda I)\beta$$
	      加入了$\lambda I$可使得$X^TX+\lambda I$可逆,故解为$\beta^*=(X^TX+\lambda I)^{-1}X^Ty$
	 \item 岭回归得到的解中加入了$\lambda I$使得$X^TX+\lambda I$可逆,是普通线性回归的改良版,计算更可靠。
	 \item $\lambda=0$时就是普通线性回归的解,$\lambda=\infty$时,正则化项的惩罚最大,故解为$\beta=0$。
	 \item 不可以。如果联合优化的话,因为$\lambda$对应项为非负的,如果限制$\lambda>0$的话,为了最小化目标函数,$\lambda$必然非常小,这样就退化为普通线性回归。如果不限制$\lambda$的话,为了最小化目标函数,$\lambda$会优化成$-\infty$,没有实际意义。所以不可以在训练集上联合优化$\lambda$和$\beta$。
\end{enumerate}
\section*{Problem 3}
\begin{enumerate}[(a)]
    \item 将$F_\beta$展开得到$$F_\beta=\frac{(1+\beta^2)TP}{(1+\beta^2)TP+\beta^2 FN+FP}$$由于分子分母各项都非负且$(1+\beta^2)TP\leq (1+\beta^2)TP+\beta^2 FN+FP$,所以$0\leq F_\beta\leq 1$,$F_\beta=0$取得条件为$TP=0$,$F_\beta=1$取得条件为$FN=FP=0$。
    \item $F_\beta$对查全率$r$和查准率$p$求偏导得$$\frac{\partial F_\beta}{\partial r}=\frac{\beta^2p^2}{(\beta^2p+r)^2},\quad\frac{\partial F_\beta}{\partial p}=\frac{r^2}{(\beta^2p+r)^2}$$
    可计算$\frac{\partial F_\beta}{\partial r}/\frac{\partial F_\beta}{\partial p}=\beta^2\frac{p^2}{r^2}$,容易看出$\beta>1$时,$\beta^2>1$,查全率更重要,$0\leq\beta<1$时,$0\leq\beta^2<1$,查准率更重要。
\end{enumerate}

\section*{Problem 4}
\begin{enumerate}[(a)]
	\item $p(x)=\sum_{y}p(x,y)=\sum_{y}p(x|y)p(y)=(\frac{1}{\sqrt{2\pi}})(e^{-2(x+1)^2}+e^{-2(x-1)^2})$
	\item $\mathbb{E}[c_{y,f(x)}]=\sum_{x,y}c_{y,f(x)}p(x,y)=\sum_{x}\sum_{y}c_{y,f(x)}p(y|x)p(x)=\sum_{x}p(x)\sum_{y}c_{y,f(x)}p(y|x)$,对于每个$x$,$\sum_{y}c_{y,f(x)}p(y|x)$是独立的,所以只需最小化$\sum_{y}c_{y,f(x)}p(y|x)$。可将其展开:$c_{1,f(x)}p(y=1|x)+c_{2,f(x)}p(y=2|x)=\mathbb{I}[f(x)\neq1]p(y=1|x)+\mathbb{I}[f(x)\neq2]p(y=2|x)$,为了使其最小,只需要比较$p(y=1|x)$与$p(y=2|x)$的大小,即$f(x)=\arg\max_y p(y|x)$,这样能使得上面式子退化为较小的一项。多分类时,此规则仍为最优。代价可写为$\mathbb{I}[f(x)\neq1]p(y=1|x)+\cdots+\mathbb{I}[f(x)\neq C]p(y=C|x)$,同样地$f(x)=\arg\max_y p(y|x)$能使得上式退化为最小的$C-1$项的和,因而是最优的。
	\item 由贝叶斯公式容易得到$p(y|x)\propto p(x|y)p(y)$,因此最优分类策略为$f(x)=\arg\max_y p(y|x)=\arg\max_y p(x|y)p(y)$,$p(x|y)$和$p(y)$均可由题中条件写出。
	\item 期望代价可写为$\mathbb{I}[f(x)\neq1]10p(y=1|x)+\mathbb{I}[f(x)\neq2]p(y=2|x)$,所以最优决策$f(x)$为$\max(10p(y=1|x), p(y=2|x)$中对应$y$的取值。
\end{enumerate}

\section*{Problem 5}
\begin{enumerate}[(a)]
	\item 由于$U^TU=I,V^TV=I$,所以$XX^T=U\Sigma V^TV\Sigma^TU^T=U\Sigma\Sigma^TU^T=U\Sigma\Sigma^TU^{-1}$,所以其特征值为$\Sigma\Sigma^T$的对角线元素,设$r=min(m,n)$,则其特征值为$\sigma_1^2,\cdots,\sigma_r^2,0,\cdots,0$,m-r个0,特征向量为$U$的对应列向量。
    \item 类似于(a),$X^TX=V\Sigma^TU^TU\Sigma V^T=V\Sigma^T\Sigma V^T=V\Sigma^T\Sigma V^{-1}$,所以其特征值为$\Sigma^T\Sigma$的对角线元素,其特征值为$\sigma_1^2,\cdots,\sigma_r^2,0,\cdots,0$,$n-r$个$0$,特征向量为$V$的对应列向量。
    \item 有相同的$r$个特征值,其为$\sigma_1,\dots,\sigma_r$的平方。
    \item $XX^T$与$X^TX$的r个特征值恰好是$X$的$r$个奇异值的平方。
    \item 我会先计算$XX^T$的特征值,因为这个矩阵维度仅为$2\times 2$,而$X^TX$的特征值就是$XX^T$的特征值再补上$99998$个$0$。
\end{enumerate}

\section*{Problem 6}
\begin{enumerate}[(a)]
    \item 忘记减去平均向量时,第一个特征向量和平均向量之间的corr小于1,较低,减去平均向量后,第一个特征向量和平均向量之间的corr较高,等于1。
    \item scale变量取值变化时,$e1$会变化,而new\_e1不变,即减去平均向量后,第一个特征向量不随scale变量的变化而变化,正确的特征向量是$$(-0.4158,0.3331,-0.7253,-0.1940,-0.1857,0.1073,0.1481,-0.1735,0.2108,-0.0994)$$。
\end{enumerate}
\end{document}